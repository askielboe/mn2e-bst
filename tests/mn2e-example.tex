% mn2e-example.tex
%
% This is a skeleton MN-format article which displays the generated
% test bibliography (mn2e-test.bbl) as it would appear in an article.
% To see this, try:
%
%     make mn2e-example.pdf
%
% Note that the mn2e-test.aux file is written by hand; the
% mn2e-example.aux file generated from this file is ignored.

\documentclass[useAMS,usenatbib]{mn2e}

\title[Illustrations of Citations]{Illustrations of Citations}
\author[Albert One and Benedict Two]{Albert One$^{1}$\thanks{E-mail:
ao@example.ac.uk (AO)} and Benedict Two$^{2}$\\
$^{1}$A Place, Somewhere\\
$^{2}$Elsewhere, On Earth}

% The mn2e class file should either import the hyperref package, or else
% provide suitable replacements for \href and \url.
%
% So, one of...
%    \usepackage{hyperref}
%
% or, in the document...
\def\href#1#2{#2}
\def\urlinner#1{#1\endgroup}
\def\url{\begingroup\def\do##1{\catcode`##1 12 }%
  \do\\\do\$\do\&\do\#\do\^\do\_\do\%\do\~ \ttfamily \urlinner}
%
% or, in the .cls file, and after the content of mn2e-bst.sty...
%    \def\href#1#2{#2}
%    \def\@url#1{#1\endgroup}
%    \def\url{\begingroup\@urlcharsother \ttfamily \@url}


\begin{document}

\date{Whenever}

\pagerange{\pageref{firstpage}--\pageref{lastpage}} \pubyear{2002}

\maketitle

\label{firstpage}

\begin{abstract}
Stuff.
\end{abstract}

\begin{keywords}
keywords
\end{keywords}

% Tests of \eprint
%\eprint{}{arXiv:yymm.1234} %-> \href{http://arxiv.org/abs/yymm.1234}{arXiv:yymm.1234}
%\eprint{}{yymm.1234} %-> same as \eprint{}{arXiv:yymm.1234}
%\eprint{arXiv}{arXiv:yymm.1234} %-> same
%\eprint{dblp}{1234} %-> \href{http://dblp.uni-trier.de/rec/bibtex/1234.xml}{dblp:1234}
%\eprint{dblp}{arXiv:yymm.1234} %-> same as \eprint{}{arXiv:yymm.1234}
%\eprint{}{wibble:1234} %-> wibble:1234 (doesn't match anything)


\section{Introduction}

There are numerous citations here.

In particular, there is
`one'~\citep{one},
`oneplus'~\citep{oneplus},
`onereprised'~\citep{onereprised},
`two'~\citep{two},
`twoplus'~\citep{twoplus},
`twobis'~\citep{twobis},
`twobook'~\citep{twobook},
`twomisc~\citep{twomisc},
`tworeprised'~\citep{tworeprised},
`three'~\citep{three},
`threeplus'~\citep{threeplus},
`threereprised'~\citep{threereprised},
`four'~\citep{four},
`fourplus'~\citep{fourplus},
`seven'~\citep{seven},
`sevenplus'~\citep{sevenplus},
`eight'~\citep{eight},
`eightplus'~\citep{eightplus},
`nine'~\citep{nine},
`nineplus'~\citep{nineplus},
`ten'~\citep{ten},
`tenplus'~\citep{tenplus} and
`tenbis'~\citep{tenbis}.

And secondly, there is
`one'~\citep{one},
`oneplus'~\citep{oneplus},
`onereprised'~\citep{onereprised},
`two'~\citep{two},
`twoplus'~\citep{twoplus},
`twobis'~\citep{twobis},
`twobook'~\citep{twobook},
`twomisc~\citep{twomisc},
`tworeprised'~\citep{tworeprised},
`three'~\citep{three},
`threeplus'~\citep{threeplus},
`threereprised'~\citep{threereprised},
`four'~\citep{four},
`fourplus'~\citep{fourplus},
`seven'~\citep{seven},
`sevenplus'~\citep{sevenplus},
`eight'~\citep{eight},
`eightplus'~\citep{eightplus},
`nine'~\citep{nine},
`nineplus'~\citep{nineplus},
`ten'~\citep{ten},
`tenplus'~\citep{tenplus} and
`tenbis'~\citep{tenbis}.



% ../mn2e.bst can generate a \logsortkey in the output, for
% debugging.  That's generally commented out, but if you need to
% uncomment that, then uncomment this, too.
%\def\logsortkey#1{{[\tiny #1]}}
\input{mn2e-test.bbl}

\bsp

\label{lastpage}

\end{document}
