% mn2e-example.tex
%
% This is a skeleton MN-format article which displays the generated
% test bibliography (mn2e-test.bbl) as it would appear in an article.
% To see this, try:
%
%     make mn2e-example.pdf
%
% Note that the mn2e-test.aux file is written by hand; the
% mn2e-example.aux file generated from this file is ignored.

\documentclass[useAMS,usenatbib]{mn2e}

\title[Illustrations of Citations]{Illustrations of Citations}
\author[Albert One and Benedict Two]{Albert One$^{1}$\thanks{E-mail:
ao@example.ac.uk (AO)} and Benedict Two$^{2}$\\
$^{1}$A Place, Somewhere\\
$^{2}$Elsewhere, On Earth}

% mn2e class file should either import the hyperref package, or else
% provide suitable dummys for \href and \url
\usepackage{hyperref}

\begin{document}

% The following implements the three-author-hack described in
% mn2e.bst.  This should be moved to mn2e.cls at some point.
%
% This consumes a command for each such author.  It's surely possible
% to avoid this (with some constructions involving {\\#1}; see
% Appendix D cleverness), but that would verge on the arcane, and not
% be really worth it.
\makeatletter
\def\mniiiauthor#1#2#3{%
  \@ifundefined{mniiiauth@#1}
    {\global\expandafter\let\csname mniiiauth@#1\endcsname\null #2}
    {#3}}
\makeatother

\date{Whenever}

\pagerange{\pageref{firstpage}--\pageref{lastpage}} \pubyear{2002}

\maketitle

\label{firstpage}

\begin{abstract}
Stuff.
\end{abstract}

\begin{keywords}
keywords
\end{keywords}

\section{Introduction}

There are numerous citations here.

In particular, there is
`one'~\citep{one},
`oneplus'~\citep{oneplus},
`onereprised'~\citep{onereprised},
`two'~\citep{two},
`twoplus'~\citep{twoplus},
`twobis'~\citep{twobis},
`twobook'~\citep{twobook},
`twomisc~\citep{twomisc},
`tworeprised'~\citep{tworeprised},
`three'~\citep{three},
`threeplus'~\citep{threeplus},
`threereprised'~\citep{threereprised},
`four'~\citep{four},
`fourplus'~\citep{fourplus},
`seven'~\citep{seven},
`sevenplus'~\citep{sevenplus},
`eight'~\citep{eight},
`eightplus'~\citep{eightplus},
`nine'~\citep{nine},
`nineplus'~\citep{nineplus},
`ten'~\citep{ten},
`tenplus'~\citep{tenplus} and
`tenbis'~\citep{tenbis}.

And secondly, there is
`one'~\citep{one},
`oneplus'~\citep{oneplus},
`onereprised'~\citep{onereprised},
`two'~\citep{two},
`twoplus'~\citep{twoplus},
`twobis'~\citep{twobis},
`twobook'~\citep{twobook},
`twomisc~\citep{twomisc},
`tworeprised'~\citep{tworeprised},
`three'~\citep{three},
`threeplus'~\citep{threeplus},
`threereprised'~\citep{threereprised},
`four'~\citep{four},
`fourplus'~\citep{fourplus},
`seven'~\citep{seven},
`sevenplus'~\citep{sevenplus},
`eight'~\citep{eight},
`eightplus'~\citep{eightplus},
`nine'~\citep{nine},
`nineplus'~\citep{nineplus},
`ten'~\citep{ten},
`tenplus'~\citep{tenplus} and
`tenbis'~\citep{tenbis}.



% ../mn2e.bst can generate a \logsortkey in the output, for
% debugging.  That's generally commented out, but if you need to
% uncomment that, then uncomment this, too.
%\def\logsortkey#1{{[\tiny #1]}}
\input{mn2e-test.bbl}

\bsp

\label{lastpage}

\end{document}
